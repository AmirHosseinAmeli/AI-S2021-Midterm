\begin{enumerate} [label=(\alph*)]
	\item
	در روش sampling Rejection در هر مرجله از نمونه‌برداری، اگر نمونه انتخاب شده مغایر با evidence مساله باشد آن را رد می‌کنیم.\\
	باتوجه به شبکه داده شده و فرضیات، برای محاسبه احتمال رد داده نمونه‌برداری شده، احتمال متمم evidence ها یعنی احتمال رخداد $P\left( { - a} \right)$ را محاسبه می‌کنیم:
	
	$$
	P\left( { - a} \right) = 0.9
	$$
	
	بنابراین در 90 درصد حالات نمونه برداشته شده رد می‌شود.
	
	\item
	در روش weighting Likelihood برای جلوگیری از دور ریختن داده‌های نامرتبط در محاسبه Query برای هر داده یک وزن باتوجه به evidence داده شده در نظر گرفته می‌شود که این وزن از رابطه زیر محاسبه می‌شود:
	
	$$
	P(E|parents(E)) = \prod\limits_i {P({e_i}|parents({e_i}))} 
	$$
	
	که برای نمونه‌های داده شده، مقدار وزن به صورت زیر می باشد:
	
	$$
	\begin{array}{l}
		{\omega _{( + a, - b, + c, + d)}} = P( + a) \times P( + d| + c) = 0.1 \times 0.5 = 0.05\\
		{\omega _{( + a, - b, - c, + d)}} = P( + a) \times P( + d| - c) = 0.1 \times 0.8 = 0.08\\
		{\omega _{( + a, + b, - c, + d)}} = P( + a) \times P( + d| - c) = 0.1 \times 0.8 = 0.08
	\end{array}
	$$
\end{enumerate}