\begin{enumerate} [label=(\alph*)]
	\item
	در روش sampling Rejection در هر مرجله از نمونه‌برداری، اگر نمونه انتخاب شده مغایر با evidence مساله باشد آن را رد می‌کنیم.\\
	باتوجه به شبکه داده شده و فرضیات، برای محاسبه احتمال رد داده نمونه‌برداری شده، احتمال متمم evidence ها یعنی احتمال رخداد $P\left( { - a} \right)$ را محاسبه می‌کنیم:
	
	$$
	P\left( { - a} \right) = 0.9
	$$
	
	بنابراین در 90 درصد حالات نمونه برداشته شده رد می‌شود.
	
	\item
	در روش weighting Likelihood برای جلوگیری از دور ریختن داده‌های نامرتبط در محاسبه query برای هر داده یک وزن باتوجه به evidence داده شده در نظر گرفته می‌شود که این وزن از رابطه زیر محاسبه می‌شود:
	
	$$
	P(E|parents(E)) = \prod\limits_i {P({e_i}|parents({e_i}))} 
	$$
	
	که برای نمونه‌های داده شده، مقدار وزن به صورت زیر می باشد:
	
	$$
	\begin{array}{l}
		{\omega _{( + a, - b, + c, + d)}} = P( + a) \times P( + d| + c) = 0.1 \times 0.5 = 0.05\\
		{\omega _{( + a, - b, - c, + d)}} = P( + a) \times P( + d| - c) = 0.1 \times 0.8 = 0.08\\
		{\omega _{( + a, + b, - c, + d)}} = P( + a) \times P( + d| - c) = 0.1 \times 0.8 = 0.08
	\end{array}
	$$
	
	حال برای محاسبه $P\left( { - b| + a, + d} \right)$ دو مجموعه زیر را تعریف می‌کنیم:
	$$
	\begin{array}{*{20}{l}}
		{{R_1} = \left\{ { + c, - c} \right\}}\\
		{{R_2} = \left\{ {( + b, + c),( + b, - c),( - b, + c),( - b, - c)} \right\}}
	\end{array}
	$$
	سپس داریم:
	$$
	P\left( { - b| + a, + d} \right) = \frac{{P\left( { + a, - b, + d} \right)}}{{P\left( { + a, + d} \right)}} = \frac{{\sum\limits_{i = 1}^M {\sum\limits_{{r_1} \in {R_1}} {\left( {{\omega _{( + a, - b,{r_1}, + d)}} \times \mathbbm{1}\left( {{x_i} = \left( { + a, - b,{r_1}, + d} \right)} \right)} \right)} } }}{{\sum\limits_{i = 1}^M {\sum\limits_{{r_2} \in {R_2}} {\left( {{\omega _{( + a,{r_2}, + d)}} \times \mathbbm{1}\left( {{x_i} = \left( { + a,{r_2}, + d} \right)} \right)} \right)} } }}
	$$
	اگر تنها عبارت بالا نیز نوشته شده باشد نمره کامل تعلق می‌گیرد. در نهایت می‌توان با نمونه‌های داده شده مقدار احتمال را به صورت زیر محاسبه کرد:
	$$
	P\left( { - b| + a, + d} \right) = \frac{{0.05 + 0.08}}{{0.05 + 0.08 + 0.08}} = \frac{{13}}{{21}}
	$$
	
	\item
	
	برای محاسبه احتمال $P(C| + a)$ چون evidence داده شده یکی از ریشه‌های گراف(راس بدون پدر) می‌باشد. بنابراین باتوجه به فرض سوال که تعداد نمونه‌ها محدود می‌باشد، تفاوتی در نحوه عملکرد دو الگوریتم وجود ندارد. البته در مقایسه زمان اجرا،‌ الگوریتم weighting likelihood سریع‌تز از sampling gibbs می‌باشد که این مساله خواسته بررسی سوال نبود.\\
	اما برای محاسبه احتمال $P(C| + d)$ چون الگوریتم weighting likelihood تنها بر انتخاب متغیرهای پائین و downstream اثر می‌گذارد ولی الگوریتم sampling gibbs این مشکل را ندارد،‌ بهتر است از روش sampling gibbs استفاده نماییم.
	
\end{enumerate}